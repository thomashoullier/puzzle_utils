\section{Ranges}
The ranges library \cite{cppreference_ranges} is part of the standard since
C++20.

\subsection{Printing ranges}
The way to print (bounded) ranges, as usually shown in cppreference, uses
\cref{lst:ex_range_printing}.

\begin{lstlisting}[language=C++, caption={Range printing example},
                   label={lst:ex_range_printing}]
  for (auto i : std::views::iota(1,4)) {
    std::cout << i << " ";
  }
  std::cout << std::endl;
\end{lstlisting}

For the sake of conciseness, we will omit the printing function in the rest of
this document.

\subsection{Range factories}
\subsubsection{iota}
\href{https://en.cppreference.com/w/cpp/ranges/iota_view}{iota} is
\emph{a range factory that generates a sequence of elements by repeatedly
  incrementing an initial value. Can be either bounded or unbounded (infinite)}.

\begin{lstlisting}[language=C++, caption={iota example}, label={lst:ex_iota}]
  std::views::iota(1) | std::views::take(5)
  std::views::iota(1,6) \end{lstlisting}

Both evaluate to the following.

\begin{lstlisting}[language={}]
  1 2 3 4 5 \end{lstlisting}
